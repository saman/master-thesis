%%%%%%%%%
% Some commands for setting up theorem environments as provided by package 
% amsthm --- language sensitive
%%%%%%%%%
\theoremstyle{plain}
\newtheorem{definition}{Definition}[chapter]
\newtheorem{lemma}[definition]{Lemma}
\ifgerman
	\newtheorem{theorem}[definition]{Satz}
	\newtheorem{corollary}[definition]{Korollar}
	\newtheorem{example}[definition]{Beispiel}
\else
	\newtheorem{theorem}[definition]{Theorem}
	\newtheorem{corollary}[definition]{Corollary}
	\newtheorem{example}[definition]{Example}
\fi



%%%%%%%%%
% Your commands should go here...
%%%%%%%%%
\newcommand*{\eg}{e.\,g.}
\newcommand*{\ie}{i.\,e.}
\newcommand*{\cf}{c.\,f.}
\newcommand*{\etal}{et~al.}

\newacro{template}[UPB-CS-TT]{Paderborn University Computer Science thesis template}

\DeclareMathOperator{\testop}{top}


%Numbers in circles
\newcommand{\fcircone}[0]{\ding{182}}
\newcommand{\fcirctwo}[0]{\ding{183}}
\newcommand{\fcircthree}[0]{\ding{184}}
\newcommand{\fcircfour}[0]{\ding{185}}
\newcommand{\fcircfive}[0]{\ding{186}}
\newcommand{\fcircsix}[0]{\ding{187}}
\newcommand{\fcircseven}[0]{\ding{188}}
\newcommand{\fcirceight}[0]{\ding{189}}
\newcommand{\fcircnine}[0]{\ding{190}}
\newcommand{\fcircten}[0]{\ding{191}}
\newcommand{\circone}[0]{\ding{192}}
\newcommand{\circtwo}[0]{\ding{193}}
\newcommand{\circthree}[0]{\ding{194}}
\newcommand{\circfour}[0]{\ding{195}}
\newcommand{\circfive}[0]{\ding{196}}
\newcommand{\circsix}[0]{\ding{197}}
\newcommand{\circseven}[0]{\ding{198}}
\newcommand{\circeight}[0]{\ding{199}}
\newcommand{\circnine}[0]{\ding{200}}
\newcommand{\circten}[0]{\ding{201}}


% Mint command
\newminted[json]{json}{fontsize=\normalsize,bgcolor=background,linenos,frame=single,framesep=2mm,rulecolor=background}
\newminted[yaml]{yaml}{fontsize=\normalsize,bgcolor=background,linenos,frame=single,framesep=2mm,rulecolor=background}
\newminted[js]{js}{fontsize=\normalsize,bgcolor=background,linenos,frame=single,framesep=2mm,rulecolor=background}
\newminted[js2]{js}{fontsize=\normalsize,bgcolor=background,frame=single,framesep=2mm,rulecolor=background}

\BeforeBeginEnvironment{code}{\vspace{5mm}}
\AfterEndEnvironment{code}{\vspace{5mm}}

