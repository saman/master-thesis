\section{References}
\label{sec:guide:references}
\LaTeX{} allows you to insert insert clickable references to other parts of the
document into your thesis.
These references help your readers avoid tedious scrolling to get to a specific
page or searching for some Figure~17, that can be located anywhere.

The clickable references are enabled from the \emph{hyperref} package that is
loaded by default via file \mbox{thesis.tex}. 
With \emph{hyperref} loaded, \LaTeX{} puts references into the table of
contents, so the reader can jump to the right section without a hassle.
However, it is good practice to insert such references yourself wherever
needed.

For example, your thesis's introduction will typically contain a paragraph that
provides an overview of the thesis's structure.
This is where you would typically add references to the sections that you talk
about, \eg{} the foundations of your work are presented in
Chapter~\ref{ch:packages}.
The reference to Chapter~\ref{ch:packages} is invoked by using command
\verb+\ref{ch:packages}+.
Of course, you need to tell \LaTeX{} what you mean by ``ch:packages.''
For that, you put a \verb+\label{ch:packages}+ right at the start of the
chapter called ``Packages and Commands,'' like you can see in the code for this
document, also replicated in Example~\ref{ex:code_label}.

\begin{example}
	\label{ex:code_label}
	This is the code for starting a new chapter and setting a label/jump
	mark:
	\begin{verbatim}
		\chapter{Packages and Commands}
		\label{ch:packages}
	\end{verbatim}
\end{example}
The \verb+\ref+ command is also used to create references to other parts of
your thesis, \eg{} figures, tables, equations, or the code example that you
find above. 

Typically, the references are rendered as numbers, like \ref{ex:code_label},
potentially including letters if the reference points to the appendix of your
document.
However, you can tell \LaTeX{} to make other text into a reference as well.
For example, the \hyperref[ex:code_label]{code example} above can also be
referenced via \verb+\hyperref[ex:code_label]{code example}+.\footnote{Note
that with \texttt{\(\backslash\)hyperref}, the label is the optional
argument!}
If you want to point to specific equations inside an equation environment, you
should use \verb+\eqref+ instead of \verb+\ref+, so the reference renders
correctly: surrounded by a pair of parentheses.

Two other classes of references are enabled by the packages \ac{template} loads
by default: URLs and bibliographic references.
The \emph{url} package in conjunction with \emph{hyperref} makes URL clickable,
and if your PDF viewer supports it, clicking such a URL accesses the URL using
your standard web browser.
Such references are created using the \verb+\url{URL}+ command.

\paragraph{Bibliographic references.}
Bibliographic references point to specific entries in the bibliography section
of your thesis, assuming you use the default setup of this template.
However, for bibliographic references you do not manually put jump marks/labels
into your document.
The marks are instead taken from the labels provided in the \mbox{.bib} file,
as discussed in Section~\ref{sec:guide:bibliographies}.
You can create bibliographic references using the \verb+\cite{label}+ command.
The \verb+\cite+ command correctly puts bibliographic references in square
brackets. 
Note that the \verb+\cite+ command can take an optional argument that is
typically used if you want to give further information, \eg{} if you talk about
Chapter~27 of the book referenced as \verb+FancyBookRef+, you should put the
reference as \verb+\cite[Chapter~27]{FancyBookRef}+.
