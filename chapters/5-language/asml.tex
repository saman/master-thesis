After research and looking for suitable existing state modeling languages, it has been decided to create a new DSL based on run-time state migration requirements.
A Domain-specific language (DSL) can be developed from scratch, or it can be a modification or a restricted form of an existing domain-specific modeling language.

The Application State Modeling Language (ASML) is a domain-specific language  (DSL) that defines each state of any software application.
The focus of this language is on modeling states of the run-time.

An application can have multiple states.
This language allows developers to write a custom model.
Each model covers an individual state. 

As mentioned in requirements (Chapter \ref{ch:requirements}), this language supports states that are migratable.
It is not supporting the transition between states as we assume we know in which state application is. There is no need to model the transition as we know what state application is.

The advantage of having an individual model for each state is improving interoperability, allowing different domain applications to share a specific part of them.
For example, an email domain application has a calendar that can migrate data with a phone calendar application.
Also, there is no need to model the states which are not migratable. So, it reduce the time of modeling state and the amount models. 

The model in ASML is called Application State Model which is discussed in next sub chapter 5.2.

The actual value of state is Run-Time State which is discussed sub chapter 5.3.

\subsection{Language Stack}
There are so many ways to define a DSL; it can be made from scratch (e.g., developing a language base on Xtext framework) or extended or restricted versions of other languages. The ASML is introduce a syntax based on JSON and is a restricted version of JSON Schema.

\subsubsection{JSON}
JSON (JavaScript Object Notation) is a file format. It is lightweight and human-readable text to store and transmit data based on the data types of the JavaScript programming language. In the last few years, many programming languages are supporting JSON, which gained popularity among developers, and has become the primary data format for exchanging information \cite{json-schema}. JSON documents consist of attribute-value pairs, which the value can be again a JSON document and there is no limit of the nesting level. A simple JSON documents is shown in Listing \ref{lis:json}.

\lstset{
  label=lis:json, caption=A simple JSON document., 
  basicstyle=\ttfamily\footnotesize, frame=single, captionpos=b,
  xleftmargin=.15\textwidth, xrightmargin=.15\textwidth
}
\begin{lstlisting}
{
    "fullname": "John Doe",
    "email": "john@mail.upb.de"
}
\end{lstlisting}

\subsubsection{JSON Schema}
After the tremendous popularity of JSON, some scenarios could benefit from a declarative way of defining a schema for JSON documents.
A declarative schema specification would give programming languages and also developers a standardized language to specify what types of JSON documents are valid as inputs and outputs \cite{json-schema}.

At the moment JSON Schema is the only general schema language for JSON documents \cite{json-model}.
The schema can be another JSON document which defines acceptable attribute-value pairs.
For instance, the simple JSON document in Listing \ref{lis:json}, can be declare in JSON Schema document in Listing \ref{lis:json-schema}

\lstset{
  label=lis:json-schema, caption=A simple JSON Schema document., 
  basicstyle=\ttfamily\footnotesize, frame=single, captionpos=b,
  xleftmargin=.15\textwidth, xrightmargin=.15\textwidth
}
\begin{lstlisting}
{
    "properties": {
        "fullname": {
            "type": "string"
        },
        "email": {
            "type": "string",
            "format": "email"
        }
    }
}
\end{lstlisting}

JSON Schema can define a document must have several properties which they can be any of regular data-types like object, array, string, boolean, number, integer and null; and for each of these types there are different keywords that help to specify and restricting the schema. The most important attribute is "type" which its value is defining the schema. Constraints can be applied on an instance by adding validation keywords to the schema \cite{json-model}. For example in Listing \ref{lis:json-schema}, \lstinline|{"type": "string"}| specify a value with string type.

\subsection{Language Schema}
The Application State Language itself is a JSON Schema document based on JSON Schema Draft-07
\footnote{\href{https://json-schema.org/draft-07/json-schema-release-notes.html}{JSON Schema Draft-07 Release Notes
}}
and is following the same logic and syntax. 

Table \ref{tab:asml-schema} shows the schema of ASML.

\begin{table}
\begin{tabularx}{\textwidth}{|l|X|X|r|}
\hline
Property             & Description                                                                              & Value                                                                                                                   \\\hline
title                & The name of the DSL.                                                                     & Application State Modeling Language                                                                                     \\\hline
description          & The description of the DSL.                                                              & A domain specific language for enabling run-time state migration between same-purpose applications of different vendors \\\hline
version              & The version of the DSL, which can be used in for version checking for validating models. & 1.0.0                                                                                                                   \\\hline
properties           & Contains the properties which a model should have.                                       & \{"asml", "info", "properties", "required"\}                                                                            \\\hline
required             & Defining the requited items which a model must have.                                     & {[}"asml", "info", "properties"{]}                                                                                      \\\hline
additionalProperties & The DSL is not allowing developers to add additional properties to models.               & false                                                                                                                   \\\hline
definitions          & Contains the definition of the DSL items and the basic JSOM Schema specification.        & version, asml, info, contact, Schema

\\\hline
\end{tabularx}
\caption{ASML Schema}
\label{tab:asml-schema}
\end{table}


The full schema of ASML is available on its GitHub repository
\footnote{\href{https://github.com/asml-lang/asml/blob/master/schemas/schema.json}{https://github.com/asml-lang/}}.

