The Run-time State is a JSON document that contains the actual attribute/values of a state. Run-time State is validated against Application State Model. If the state is valid, it can be migrated from source to target application.
Target application should adjust itself with the new state.

\subsection{Example Run-time States}

\subsubsection{Composing New E-mail}
Based on composing new e-mail Application State Model (Listing \ref{lis:sending-email-schema}) valid example values for sending e-mail state is shown in Listing \ref{lis:sending-email-state}.

\lstset{
  label=lis:sending-email-state, caption=A Run-time State for sending e-mail as JSON document., 
}
\begin{lstlisting}[language=json]
{
  "from": "saman@mail.upb.de",
  "to": [
    "engels@uni-paderborn.de",
    "dennis.wolters@upb.de"
  ],
  "subject": "Master Thesis Documents",
  "body": "Dear Prof. Engels\n, Here is my master thesis."
}
\end{lstlisting}


\subsubsection{Writing Note}
Based on writing note Application State Model (Listing \ref{lis:note-schema}) valid example values for writing note state is shown in Listing \ref{lis:writing-note-state}.
 
\lstset{
  label=lis:writing-note-state, caption=A Run-time State for writing note as JSON document.
}
\begin{lstlisting}[language=json]
{
    "text": "ideas about the thesis"
}
\end{lstlisting}

\subsubsection{Search}
Based on search Application State Model (Listing \ref{lis:search-schema}) valid example values for search state is shown in Listing \ref{lis:search-state}.
\lstset{
  label=lis:search-state, caption=A Run-time State for search as JSON document.
}
\begin{lstlisting}[language=json]
{
    "query": "paderborn university",
    "submit": true
}
\end{lstlisting}