\chapter{Language for Run-time State Migration}
\label{ch:language}
As stated by \textbf{R2 Platform-independent State Specification}, need a platform-independent representation of run-time state. For this, we develop a language to describe run-time states. The requirements for this language are on the requirements D1 to D3 (Chapter \ref{ch:requirements}).
In this chapter, the DSL definition and interpretation, specification of run-time state, and terminology used are described.
% This includes the interpretation which defines the semantics, meta-model description which describe the abstract syntax, run-time state specification which is the definition of the concrete syntax, and some run-time state specification examples.
The definition of the language occurs on different layers (Figure \ref{fig:asml}). On the top layer, we define a language for specify a type of state an application. Such a specification of a specific type of state resides on the middle layer. Concrete instances of this type, a specific run-time state of an application, is located on the bottom layer. All three layers are explained subsequently.

\FloatBarrier
\begin{figure}[H]
    \includegraphics[scale=0.77]{../figures/asml.pdf}
    \centering
    \caption{The definition of the language on different layers.}
    \label{fig:asml}
\end{figure}
\FloatBarrier


\section{Application State Modeling Language}
After research and looking for suitable existing state modeling languages, it has been decided to create a new DSL based on run-time state migration requirements. A Domain-specific language (DSL) can be developed from scratch, or it can be a modification or a restricted form of an existing modeling language.

The Application State Modeling Language (ASML) is a DSL that defines each state of any software application. The focus of this language is on modeling states of the run-time.

An application can have multiple states. This language allows developers to write a custom model. Each model covers an individual state. 

As mentioned in requirements (Chapter \ref{ch:requirements}), this language supports states that are migratable. It is not supporting the transition between states as we assume we know in which state application is.

The advantage of having an individual model for each state is improving interoperability, allowing different domain applications to share a specific part of them. For example, an email domain application has a calendar that can migrate data with a phone calendar application.


\subsection{Language Stack}
There are so many ways to define a DSL; it can be made from scratch (e.g., developing a language base on Xtext framework) or extended or restricted versions of other languages. The ASML is introduce a syntax based of JSON and is a restricted version of JSON Schema.

\subsubsection{JSON}
JSON (JavaScript Object Notation) is a file format. It is lightweight and human-readable text to store and transmit data based on the data types of the JavaScript programming language. In the last few years, many programming languages are supporting JSON, which gained popularity among developers, and has become the primary data format for exchanging information \cite{json-schema}. JSON documents consist of attribute-value pairs, which the value can be again a JSON document and there is not limit of the nesting level. A simple JSON documents is shown in Listing \ref{lis:json}.

\lstset{
  label=lis:json, caption=A simple JSON document., 
  basicstyle=\ttfamily\footnotesize, frame=single, captionpos=b,
  xleftmargin=.15\textwidth, xrightmargin=.15\textwidth
}
\begin{lstlisting}
{
    "fullname": "John Doe",
    "email": "john@mail.upb.de"
}
\end{lstlisting}

\subsubsection{JSON Schema}
After the tremendous popularity of JSON, some scenarios could benefit from a declarative way of defining a schema for JSON documents. A declarative schema specification would give programming languages and also developers a standardized language to specify what types of JSON documents are valid as inputs and outputs \cite{json-schema}.

The schema can be another JSON document which defines acceptable attribute-value pairs. For instance, the simple JSON document in Listing \ref{lis:json}, can be declare in JSON schema document in Listing \ref{lis:json-schema}

\lstset{
  label=lis:json-schema, caption=A simple JSON schema document., 
  basicstyle=\ttfamily\footnotesize, frame=single, captionpos=b,
  xleftmargin=.15\textwidth, xrightmargin=.15\textwidth
}
\begin{lstlisting}
{
    "properties": {
        "fullname": {
            "type": "string"
        },
        "email": {
            "type": "string",
            "format": "email"
        }
    }
}
\end{lstlisting}


\subsection{Language Schema}
description of the schema of the ASML.

\begin{itemize}
\item title
\item description
\item version
\item properties
\item required
\item additionalProperties
\item definitions
\end{itemize}

\subsection{Data Types}
here are data type of the ASML.


\begin{itemize}
\item array
\item boolean
\item integer
\item number
\item object
\item string
\item null
\end{itemize}


\section{Application State Model}

The Application State Model is a JSON Schema document which defines what a state must contains at run-time and is get validated against Application State Modeling Language. Developers can write a custom specification for a run-time state by Application State Model in a JSON Schema document.

Any application have different states, Each state should has its own model which is a JSON Schema document. At the time of integration, Application State Models must be coupled with the source and target applications to define which states are available for migration.

For migration happen source and target applications must have a common Application State Model. As each state modeled individually, different domain application with a same-purpose part can migrate their states. For example, an e-mail client might have a to-do list which is common with a task management application.




\subsection{Top-level Fields}
There are some fields that they need to be in a Application State Model to define the specification of a state.

\subsubsection{asml}
A model must have an object field named "asml" which its string value represent the version of ASML is based on. This version follows SemVer pattern \footnote{\href{https://semver.org/}{https://semver.org/
}} and helps validating the model against the right version of our DSL. Listing \ref{lis:asm-asml} shows how the "asml" field has to defined.

\lstset{
  label=lis:asm-asml, caption=Application State Model "asml" field example., 
  basicstyle=\ttfamily\footnotesize, frame=single, captionpos=b,
  xleftmargin=.15\textwidth, xrightmargin=.15\textwidth
}
\begin{lstlisting}
{
    "asml": "1.0.0"
}
\end{lstlisting}
\subsubsection{info}
A model must have an object field named "info" which its object value represent the information about the model. This object must have objects fields "title" and "version" as common models are distinguished with their two string value. Other fields are "description" and "contact", which they are optional. The "description" is a string value represent the descriptive text about the purpose of the model, and "contact" is the information of the author of the model in case of further contacts. Listing \ref{lis:asm-info} shows an example of how the "info" field has to be defined.

\lstset{
  label=lis:asm-info, caption=Application State Model "info" field example., 
  basicstyle=\ttfamily\footnotesize, frame=single, captionpos=b,
  xleftmargin=.15\textwidth, xrightmargin=.15\textwidth
}
\begin{lstlisting}
{
    "title": "sending-email",
    "description": "A schema model for sending an email",
    "version": "1.0.0",
    "contact": {
        "name": "Saman Soltani",
        "email": "saman@mail.upb.de",
        "url": "samansoltani.com"
    }
}
\end{lstlisting}

\subsubsection{properties}
A model must have an object field named "properties" which its object value represent the specification of a state. Based on ASML, any element of "properties" object consider as a variable that can be migrated.
Listing \ref{lis:asm-properties} shows an example which has four fields with different types such as from, to, subject and body. These fields should be in the state.

\lstset{
  label=lis:asm-properties, caption=Application State Model "properties" field example., 
  basicstyle=\ttfamily\footnotesize, frame=single, captionpos=b,
  xleftmargin=.15\textwidth, xrightmargin=.15\textwidth
}
\begin{lstlisting}
{
    "properties": {
      "from": {
           "type": "string",
           "format": "email"
      },
      "to": {
           "type": "array",
           "items": {
                "type": "string",
                "format": "email"
           }
      },
      "subject": {
           "type": "string"
      },
      "body": {
           "type": "string"
      }
    }
}
\end{lstlisting}
\subsubsection{required}
A model can have an object field named "required" which its array of strings value represent compulsory variables must have in the state.
Listing \ref{lis:asm-required} shows an example which has three string value in an array. These fields must be exist in the state as they are specified in "required" field.

\lstset{
  label=lis:asm-required, caption=Application State Model "required" field example., 
  basicstyle=\ttfamily\footnotesize, frame=single, captionpos=b,
  xleftmargin=.15\textwidth, xrightmargin=.15\textwidth
}
\begin{lstlisting}
{
    "required": ["from", "to", "body"]
}
\end{lstlisting}

\subsection{Example Models}
\subsubsection{Composing New E-mail}
Considering an e-mail client application which user can write an e-mail. The state of composing a new e-mail can have minimum four fields which they are from, to, subject and body. A combination of Listings \ref{lis:asm-asml},  \ref{lis:asm-info},  \ref{lis:asm-properties} and  \ref{lis:asm-required} shows an example of this model.
The complete version of Application State Model of composing a new e-mail is shown in Listing \ref{lis:sending-email-schema}.
\subsubsection{Writing Note}
Considering a note taking application which has only one state for migration which is writing note. It can be modeled with a compulsory "text" string field as in Listing \ref{lis:note-schema}.

\lstset{
  label=lis:note-schema, caption=Note Writing example model as JSON Schema document., 
  basicstyle=\ttfamily\footnotesize, frame=single, captionpos=b,
  xleftmargin=.01\textwidth, xrightmargin=.01\textwidth,
  breaklines=true
}
\begin{lstlisting}
{
    "asml": "1.0.0",
    "info": {
        "title": "writing-note",
        "version": "1.0.0"
    },
    "properties": {
        "text": {
            "description": "content of the note",
            "type": "string"
        }
    },
    "required": [ "text" ]
}
\end{lstlisting}
\subsubsection{Search}
Searching between data is very common between applications. If there would a search feature; Also, there is a search result. The search state can be modeled in a way that application knows if the search has been submitted by the user or not; So it can be adjust itself base on a received state. The search model can have a compulsory "query" string field which defines the text that user is looking for, and "submit" boolean field which shows if the search has been submitted.

This Application State Model can be used in different purpose applications which they have the search ability. Listing \ref{lis:search-schema} shows a search state modeled defined as Application State Model in a JSON Schema document.

\lstset{
  label=lis:search-schema, caption=Search model as JSON Schema document., 
  basicstyle=\ttfamily\footnotesize, frame=single, captionpos=b,
  xleftmargin=.01\textwidth, xrightmargin=.01\textwidth,
  breaklines=true
}
\begin{lstlisting}
{
    "asml": "1.0.0",
    "info": {
        "title": "search",
        "version": "1.0.0"
    },
    "properties": {
        "query": {
            "description": "the query of search",
            "type": "string"
        },
        "submit": {
            "description": "shows if the query is already has been submitted",
            "type": "boolean"
        }
    },
    "required": [ "query" ]
}
\end{lstlisting}

\section{Run-time State}
The Run-time State is a JSON document that contains the actual key-value pairs of a state. Run-time State is validated against Application State Model. If the state is valid, it can be migrated from source to target application.
Target application should adjust itself with the new state.

\subsection{Example Run-time States}

\subsubsection{Composing New E-mail}
Based on composing new e-mail Application State Model (Listing \ref{lis:sending-email-schema}) valid example values for sending e-mail state is shown in Listing \ref{lis:sending-email-state}.


\FloatBarrier
\begin{code}
\begin{json}
{
  "from": "saman@mail.upb.de",
  "to": [
    "engels@uni-paderborn.de",
    "dennis.wolters@upb.de"
  ],
  "subject": "Master Thesis Documents",
  "body": "Dear Prof. Engels\n, Here is my master thesis."
}
\end{json}
\caption{A Run-time State for sending e-mail as JSON document.}
\label{lis:sending-email-state}
\end{code}
\FloatBarrier

\subsubsection{Writing Note}
Based on writing note Application State Model (Listing \ref{lis:note-schema}) valid example values for writing note state is shown in Listing \ref{lis:writing-note-state}.
 
\FloatBarrier
\begin{code}
\begin{json}
{
    "text": "ideas about the thesis"
}
\end{json}
\caption{A Run-time State for writing note as JSON document}
\label{lis:writing-note-state}
\end{code}
\FloatBarrier


\subsubsection{Search}
Based on search Application State Model (Listing \ref{lis:search-schema}) valid example values for search state is shown in Listing \ref{lis:search-state}.

\FloatBarrier
\begin{code}
\begin{json}
{
    "query": "paderborn university",
    "submit": true
}
\end{json}
\caption{A Run-time State for search as JSON document.}
\label{lis:search-state}
\end{code}
\FloatBarrier