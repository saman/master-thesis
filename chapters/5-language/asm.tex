
The Application State Model is a custom JSON Schema document that defines what a state must contain at run-time and is validated against Application State Modeling Language. To enable run-time state migration, developers should write a custom Application State Model or use an existing one. In contrast to a regular JSON schema, the Application State Model must have the fields defined by the ASML.

Any application has different states; Each state should have its own model, which is a JSON Schema document. At the time of integration, Application State Models must be coupled with the source and target applications to determine which states are available for migration.

For migration to happen, source and target applications must have a common Application State Model. As each state is modeled individually, different domain application with the same-purpose part can migrate their states. For example, an e-mail client may have a to-do list feature which can be a standard characteristic for a task management application. So, they may have some common states like writing a task or search.

Developers can write their own Application State Model or using the existing ones. Variables of a run-time state that need to be migrated shall be identified and associated to keywords. Developers can look into the Model Repository (explained in 6.5) and search for these keywords to find a suitable model for their application. They may use an existing one or even extending it by making a new model. To write a new model, variables of a run-time state which they are associated to some phrases, should be specified by their name, type or their format. For example, when a search state needs to be migrated, we need to migrate a text and maybe the submission status. These variables can be associate with some phrases like “query” as a string and “submit” as a boolean. These associated phrases and their type must be specified in \lstinline[basicstyle=\ttfamily]{properties} section of Application State Model which is described in next section. 

\subsection{Top-level Fields}
There are some fields that they need to be in an Application State Model to define a state specification.

\subsubsection{asml}
As stated in requirement \textbf{D3 Validating}, a model must have an object field named “asml” whose string value represents the version of ASML. This version follows SemVer pattern \footnote{\url{https://semver.org/}} and helps to validate the model against the correct version of our DSL. Listing \ref{lis:asm-asml} shows how the “asml” field has to be defined. The version of ASML in this Application State Model is “1.0.0”.

\lstset{
  label=lis:asm-asml, caption=Application State Model “asml” field example., 
}
\begin{lstlisting}[language=yaml]
asml: 1.0.0
\end{lstlisting}
\subsubsection{info}
As stated in requirement \textbf{D2 Finding Same Model}, a model must have an object field named “info” whose object value represents the basic model’s information. This object must have two string values “title” and “version”. Also, “keywords” is an array of string which should contains some keywords that other developers can find this model in Model Repository. Common models can be fined and distinguished by values of “title”, “version” and “keywords”. Other fields are “description” and “contact”, which are optional. The “description” is a string value that represents the descriptive text about the purpose of the model, and “contact” is the author’s information in case of further contacts. Listing \ref{lis:asm-info} shows an example of how the “info” field has to be defined.

\lstset{
  label=lis:asm-info, caption=Application State Model “info” field example. 
}
\begin{lstlisting}[language=yaml]
asml: 1.0.0
info:
  title: sending-email
  description: A schema model for run-time state migration of sending an email
  version: 1.0.0
  keywords:
  - email
  - e-mail
  - compose  
  contact:
    name: Saman Soltani
    email: saman@mail.upb.de
    url: samansoltani.com
\end{lstlisting}

\subsubsection{properties}
A model must have an object field named “properties” whose object value represents a state specification. Based on ASML, any element of the “properties” object is considered a variable that can be migrated.
Listing \ref{lis:asm-properties} shows an example that has four fields with different types such as from, to, subject and body. These fields should be in the state.

\lstset{
  label=lis:asm-properties, caption=Application State Model “properties” field example.
}
\begin{lstlisting}[language=yaml]
properties:
  from:
    type: string
    format: email
  to:
    type: array
    items:
      type: string
      format: email
  subject:
    type: string
  body:
    type: string

\end{lstlisting}
\subsubsection{required}
As stated in requirement \textbf{D1 State Specification}, a model can have an object field named “required” which its array of strings value represents compulsory variables in the state.
Listing \ref{lis:asm-required} shows an example which has three string value in an array. These fields must exist in the state as they are specified in “required” field.

\lstset{
  label=lis:asm-required, caption=Application State Model “required” field example.
}
\begin{lstlisting}[language=yaml]
required:
- from
- to
- body

\end{lstlisting}

\subsection{Example Models}
\subsubsection{Composing New E-mail}
Considering an e-mail client application in which the user can write an e-mail. The state of composing a new e-mail can have a minimum of four fields which are “from”, “to”, “subject” and “body”. A combination of Listings \ref{lis:asm-asml},  \ref{lis:asm-info},  \ref{lis:asm-properties} and  \ref{lis:asm-required} shows an example of this model.
The complete version of the Application State Model of composing a new e-mail is shown in  Listing \ref{lis:sending-email-schema} which is in appendixes.
\subsubsection{Writing Note}
Considering a note-taking application that has only one state for migration which is writing a note. It can be modeled with a compulsory “text” string field as in Listing \ref{lis:note-schema}.

\lstset{
  label=lis:note-schema, caption=Note Writing example Application State Model as JSON Schema in YAML.
}
\begin{lstlisting}[language=yaml]
asml: 1.0.0
info:
  title: writing-note
  version: 1.0.0
properties:
  text:
    description: content of the note
    type: string
required:
- text

\end{lstlisting}
\subsubsection{Search}
Searching between data is very common between applications. Suppose an application includes a search function. It also provides a result for searches. The search state can be modeled so that the application knows if the search has been submitted by the user or not; it can adjust itself base on a received state. The search model can have a compulsory “query” string field, which defines the text that the user is looking for, and a “submit” boolean field shows if the search has been submitted.

This Application State Model can be used in different purpose applications which they have the search feature. Listing \ref{lis:search-schema} shows a search state modeled defined as Application State Model in a JSON Schema document.

\lstset{
  label=lis:search-schema, caption=Search example Application State Model as JSON Schema in YAML.
}
\begin{lstlisting}[language=yaml]
asml: 1.0.0
info:
  title: search
  version: 1.0.0
properties:
  query:
    description: the query of search
    type: string
  submit:
    description: shows if the query is already has been submitted
    type: boolean
required:
- query
\end{lstlisting}
