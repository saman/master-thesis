
The Application State Model is a JSON Schema document which defines what a state must contains at run-time and is get validated against Application State Modeling Language. Developers can write a custom specification for a run-time state by Application State Model in a JSON Schema document.

Any application have different states, Each state should has its own model which is a JSON Schema document. At the time of integration, Application State Models must be coupled with the source and target applications to define which states are available for migration.

For migration happen source and target applications must have a common Application State Model. As each state modeled individually, different domain application with a same-purpose part can migrate their states. For example, an e-mail client might have a to-do list which is common with a task management application.




\subsection{Top-level Fields}
There are some fields that they need to be in a Application State Model to define the specification of a state.

\subsubsection{asml}
A model must have an object field named "asml" which its string value represent the version of ASML is based on. This version follows SemVer pattern \footnote{\href{https://semver.org/}{https://semver.org/
}} and helps validating the model against the right version of our DSL. Listing \ref{lis:asm-asml} shows how the "asml" field has to defined.

\lstset{
  label=lis:asm-asml, caption=Application State Model "asml" field example., 
  basicstyle=\ttfamily\footnotesize, frame=single, captionpos=b,
  xleftmargin=.15\textwidth, xrightmargin=.15\textwidth
}
\begin{lstlisting}
{
    "asml": "1.0.0"
}
\end{lstlisting}
\subsubsection{info}
A model must have an object field named "info" which its object value represent the information about the model. This object must have objects fields "title" and "version" as common models are distinguished with their two string value. Other fields are "description" and "contact", which they are optional. The "description" is a string value represent the descriptive text about the purpose of the model, and "contact" is the information of the author of the model in case of further contacts. Listing \ref{lis:asm-info} shows an example of how the "info" field has to be defined.

\lstset{
  label=lis:asm-info, caption=Application State Model "info" field example., 
  basicstyle=\ttfamily\footnotesize, frame=single, captionpos=b,
  xleftmargin=.15\textwidth, xrightmargin=.15\textwidth
}
\begin{lstlisting}
{
    "title": "sending-email",
    "description": "A schema model for sending an email",
    "version": "1.0.0",
    "contact": {
        "name": "Saman Soltani",
        "email": "saman@mail.upb.de",
        "url": "samansoltani.com"
    }
}
\end{lstlisting}

\subsubsection{properties}
A model must have an object field named "properties" which its object value represent the specification of a state. Based on ASML, any element of "properties" object consider as a variable that can be migrated.
Listing \ref{lis:asm-properties} shows an example which has four fields with different types such as from, to, subject and body. These fields should be in the state.

\lstset{
  label=lis:asm-properties, caption=Application State Model "properties" field example., 
  basicstyle=\ttfamily\footnotesize, frame=single, captionpos=b,
  xleftmargin=.15\textwidth, xrightmargin=.15\textwidth
}
\begin{lstlisting}
{
    "properties": {
      "from": {
           "type": "string",
           "format": "email"
      },
      "to": {
           "type": "array",
           "items": {
                "type": "string",
                "format": "email"
           }
      },
      "subject": {
           "type": "string"
      },
      "body": {
           "type": "string"
      }
    }
}
\end{lstlisting}
\subsubsection{required}
A model can have an object field named "required" which its array of strings value represent compulsory variables must have in the state.
Listing \ref{lis:asm-required} shows an example which has three string value in an array. These fields must be exist in the state as they are specified in "required" field.

\lstset{
  label=lis:asm-required, caption=Application State Model "required" field example., 
  basicstyle=\ttfamily\footnotesize, frame=single, captionpos=b,
  xleftmargin=.15\textwidth, xrightmargin=.15\textwidth
}
\begin{lstlisting}
{
    "required": ["from", "to", "body"]
}
\end{lstlisting}

\subsection{Example Models}
\subsubsection{Composing New E-mail}
Considering an e-mail client application which user can write an e-mail. The state of composing a new e-mail can have minimum four fields which they are from, to, subject and body. A combination of Listings \ref{lis:asm-asml},  \ref{lis:asm-info},  \ref{lis:asm-properties} and  \ref{lis:asm-required} shows an example of this model.
The complete version of Application State Model of composing a new e-mail is shown in Listing \ref{lis:sending-email-schema}.
\subsubsection{Writing Note}
Considering a note taking application which has only one state for migration which is writing note. It can be modeled with a compulsory "text" string field as in Listing \ref{lis:note-schema}.

\lstset{
  label=lis:note-schema, caption=Note Writing example model as JSON Schema document., 
  basicstyle=\ttfamily\footnotesize, frame=single, captionpos=b,
  xleftmargin=.01\textwidth, xrightmargin=.01\textwidth,
  breaklines=true
}
\begin{lstlisting}
{
    "asml": "1.0.0",
    "info": {
        "title": "writing-note",
        "version": "1.0.0"
    },
    "properties": {
        "text": {
            "description": "content of the note",
            "type": "string"
        }
    },
    "required": [ "text" ]
}
\end{lstlisting}
\subsubsection{Search}
Searching between data is very common between applications. If there would a search feature; Also, there is a search result. The search state can be modeled in a way that application knows if the search has been submitted by the user or not; So it can be adjust itself base on a received state. The search model can have a compulsory "query" string field which defines the text that user is looking for, and "submit" boolean field which shows if the search has been submitted.

This Application State Model can be used in different purpose applications which they have the search ability. Listing \ref{lis:search-schema} shows a search state modeled defined as Application State Model in a JSON Schema document.

\lstset{
  label=lis:search-schema, caption=Search model as JSON Schema document., 
  basicstyle=\ttfamily\footnotesize, frame=single, captionpos=b,
  xleftmargin=.01\textwidth, xrightmargin=.01\textwidth,
  breaklines=true
}
\begin{lstlisting}
{
    "asml": "1.0.0",
    "info": {
        "title": "search",
        "version": "1.0.0"
    },
    "properties": {
        "query": {
            "description": "the query of search",
            "type": "string"
        },
        "submit": {
            "description": "shows if the query is already has been submitted",
            "type": "boolean"
        }
    },
    "required": [ "query" ]
}
\end{lstlisting}