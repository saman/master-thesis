\section{Bibliographies}
\label{sec:guide:bibliographies}

Typically you do not want to manage you bibliographic references yourself.
\LaTeX{} and its helper program \emph{bibtex} manage them for you.
You just need to tell them what references you are going to use.
You do that by putting a bibliographic reference --- called \emph{bib entry}
--- into file \mbox{bibliography.bib}. 

The bib entries are as shown in Example~\ref{ex:bib_entry}.
The key=value pairs depend on the type of the bib entry.
Certain keys must be present in an entry of a given type, while others can be
missing.
\begin{example}
\label{ex:bib_entry}
This is an example bib entry of type ``inproceedings.'' It starts with an @
symbol, immediately followed by its type and an openeing curly brace and a
label (LABEL in this example) and a comma.
What follows is a comma-separated list of key=value pairs, with the values
being surrounded by curly braces.
The bib entry closes with a final curly brace.
\begin{verbatim}
@inproceedings{LABEL,
  author    = {First Author and Second Author and ...},
  editor    = {A List of Fancy People},
  title     = {Example Title},
  booktitle = {Proceedings of Conference 1468},
  pages     = {123--456},
  publisher = {Publisher's Name},
  year      = {1468},
}
\end{verbatim}
Note that you can enforce a certain type of letter capitalization, \ie{} all
caps, by putting the respective term inside an extra pair of curly braces,
\eg{} \verb+publisher = {ACM},+ may be rendered as Acm, while 
\verb+publiser = {{ACM}},+ is always rendered as ACM.
\end{example}

Using \emph{bibtex}, you can compile a nicely looking list of bibliographic
references.
The list is included via the \emph{biblatex} package at the place indicated by
the \verb+\printbibliography+ command.
You can refer to the individual list entries by using the label (in the example
LABEL) of the corresponding bib entry in a \verb+\cite+ command as described in
Section~\ref{sec:guide:references}.
Hence, it is useful to derive a bib entry's label from the information provided
by the entry itself, \eg{} the title.

While \emph{bibtex} provides means to manage your bibliographic references,
they do not help you with obtaining all relevant, and particularly complete,
bib entries. 
You should create bib entries yourself, unless all other options fail.
Typically, publishers of computer science literature provide bib entries on
their web pages.
Also Google Scholar\footnote{\url{https://scholar.google.com}} provides bib
entries.
However, both sources sometimes provide very incomplete bib entries.
Complete entries can typically obtained from
DBLP.\footnote{\url{http://dblp.org/search/index.php}}

