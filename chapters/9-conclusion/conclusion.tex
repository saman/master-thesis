\chapter{Conclusion and Future Work}
\label{ch:conclusion}

Conclusion and Future Work
\section{Conclusion}
This thesis presents an approach for enabling run-time state migration between same-purpose applications of different vendors. We investigate the problem by analyzing same-purpose applications and draft some requirements. We suggest a Domain-specific language for modeling run-time states based on JSON Schema. The name of this DSL is Application State Modeling Language (ASML) and allows developers to write a platform-independent state specification for each state. This state specification is the Application State Model. The ASML allows us to validate the Application State Model against its schema. Also, a run-time state can be validated against Application State Model.
Moreover, we present a repository manager for Application State Models, called Model Repository. This repository allows developers to find and select an existing Application State Model. Also, they can write their model and contribute by adding it to the repository. Furthermore, we discuss the architecture that we use for implementation. We use the publish-subscribe pattern as the main architecture and discuss the role of middleware as a message broker and its messages, actions, and topics. To enable other developers, we decide to ease the work by having libraries. These libraries allow developers to write some glue code and user interfaces and implement life cycles that we discuses for enabling run-time state for existing applications. We implement the middleware with MQTT Protocol with Mosquitto message broker. Two libraries in JavaScript and Java have been developed based on our API Reference. For testing purposes with integrating these libraries in two demo applications.
Additionally, we develop some helpers tools to help developers and make their job easier. Finally, we implemented our approach on real-world example existing applications. We integrate libraries in Mailspring and K-9 Mail and change their UI to adapt to the approach.

\newpage
\section {Future Work}
In the future, we want to tackle the following points. 
\paragraph{File Transfer Support}
Firstly, we want to add the support of file transfer. Currently, our implementation only supports primitive data types.

\paragraph{Security}
Secondly, we want to add security to the commutation between the library and middleware. At the moment, the run-time states are not encrypted. Thereby, it can be end-to-end encryption between devices.

\paragraph{Permissions}
Thirdly, we want to add permissions to our implementation. For example, two different mailboxes should not be able to migrate their run-time states.

\paragraph{Supporting Different Model Version}
Fourthly, we want to add the support of distinguishing different versions of the Application State Model and run-time state. Currently, our implementation accepts any version of the Application State Model and run-time states.

\paragraph{Refine the Modeling Approach}
Finally, we want to support run-time state migration between applications for which no common Application State Model can be found. We want to refine the modeling approach to cope with differences in an Application State Model, e.g., by using model transformations.

