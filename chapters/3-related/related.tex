\chapter{Related Work}
In this section, we discuss the state of the art in run-time state migration.
\label{ch:related}

Various articles have been published regarding state migration, mostly focusing on run-time state migration of one particular platform to another device. For instance, In \cite{r1-10.1145/1851600.1851653}, the researchers presents an approach that can migrate part of a web application with its state across devices. Users need to use the developed tool called \textit{Partial Migration} to select elements of a web page. The list of these elements are auto-generated from HTML and CSS tags in a hierarchical tree view. The user can choose a target device from provided list called \textit{Migration Panel} which contains devices hostname by a figure highlighting the type of the corresponding device (desktop, PDA, smartphone, large screen, ..). The user can start the migration process by clicking on \textit{MIGRATE} button. The target device receives a migration request with the information of the source device. If the user accept the migration request, the target device shows a web page which contains the chosen elements with their state. This method use user interface description languages \cite{r5-maria-10.1145/1614390.1614394} and tools that can automatically reverse engineer existing desktop Web content in order to build the corresponding logical description in a hierarchical tree. As the main developed tools is a desktop application, in this approach the source application always should be a desktop web application which its front-end is implemented using HTML, CSS, and JavaScript.

Another migration approach focusing on web applications is discuss in \cite{r2-zaplata}. The researchers presented Panelrama, a web framework for cross-device UI distribution, and a developer study showing its advantages. Their implementation enables developers to migrate at a relatively low cost through built-in mechanisms for synchronizing UI states, which do not require drastic changes to existing languages. Developers can use divided groups called \textit{panels} in Panelrama to separate the user interface of an application into different groups. Panelrama gives developers the opportunity to score the importance of certain device characteristics to this panel's usability. Applications with multi-device support can distribute panels to best fit the devices for an optimal user experience, such as video to the largest device and remote controls for the closer devices. This framework is only targeting the web and hybrid applications which are based on HTML5.

The paper \cite{r4-5392926} describes ScudOSGi, a framework for handling task migration that enables facility-involved task migration in the OSGi which is a Java framework.
They use a whiteboard model to simplify the state’s management and maintenance.
This framework has four components which are task migration trigger and task host, server, and local infrastructure. 
The server manages the state and resources of user tasks.
Computation and facilities services in the local space are available to tasks and are managed by local infrastructure.
In order to allow local infrastructure to determine whether the task requires certain facilities, the task migration trigger parses both the task and the application description.
The task host hosts the applications needed by the task, and it's responsible for the communication channel required to get information from the Server about the task's state and resources.
An application model with two levels is proposed by them. 
Compared with other migration frameworks, they focus on maximizing usage of local facilities for user tasks.
As examples for applications of ScudOSGi, they investigated two use-cases: smart player and multi-user shopping.

