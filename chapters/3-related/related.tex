\chapter{Related Works}
In this section, we discuss state of the art in run-time state migration.
\label{ch:related}

Various articles have been published regarding state migration, mainly focusing on run-time state migration of one particular platform to another device. For instance, in \cite{r1-10.1145/1851600.1851653}, the researchers present an approach that can migrate part of a web application with its state across devices. Users need to use the developed tool called \textit{Partial Migration} to select elements of a web page. The list of these elements is auto-generated from HTML and CSS tags in a hierarchical tree view. The user can choose a target device from the provided list called \textit{Migration Panel} which contains the device’s hostname by an icon highlighting the type of the corresponding device (desktop, PDA, smartphone, large screen). She can start the migration process by clicking on \textit{MIGRATE} button. The target device receives a migration request with the information of the source device. If she accepts the migration request, the target device shows a web page that contains the chosen elements with their state. This method uses a user interface description languages \cite{r5-maria-10.1145/1614390.1614394} and tools that can automatically reverse engineer existing desktop Web content to build the corresponding logical description in a hierarchical tree. As the primary developed tool is a desktop application, in this approach, the source application should always be a desktop web application which its front-end is implemented using HTML, CSS, and JavaScript.

Another migration approach focusing on web applications is discussed in \cite{r2-zaplata}. The researchers presented Panelrama, a web framework for cross-device UI distribution. Their implementation enables developers to migrate at a relatively low cost through built-in mechanisms for synchronizing UI states, which do not require drastic changes to existing languages. Developers can use divided groups called \textit{panels} in Panelrama to separate the user interface of an application into different groups. Panelrama gives developers the opportunity to score the importance of certain device characteristics to this panel’s usability. Applications with multi-device support can distribute panels to best fit the devices for an optimal user experience, such as video to the largest device and remote controls for the closer devices. This framework is only targeting the web and hybrid applications that are based on HTML5.

The paper \cite{r6-10.1145/2254556.2254563} also suggests how to migrate interactive web applications to users with multiple devices.
Regardless of the devices being used (desktops, tablets, smartphones), the environment enables dynamic push and pull of interactive Web applications across desktop and mobile devices while preserving their state.
Users may choose to migrate to a specific device using a migration client by accessing it through a browser.
This approach only works on web applications, and its needs a browser to operate.

All of these approaches are oriented towards web applications, whereas our approach is also applicable to desktop and mobile applications.

Furthermore, several studies have discussed different aspects of run-time migration in a software system. The paper \cite{r4-5392926} describes ScudOSGi, a framework for handling task migration that enables facility-involved task migration in the OSGi which is a Java framework for component-based development and dependency management. With ScudOSGi they allow to migrate those components.
They use a whiteboard model to simplify the state’s management and maintenance.
This framework has four components which are task migration trigger and task host, server, and local infrastructure. 
The server manages the state and resources of user tasks.
Computation and facilities services in the local space are available to tasks and are managed by local infrastructure.
In order to allow local infrastructure to determine whether the task requires certain facilities, the task migration trigger parses both the task and the application description.
The task host hosts the applications needed by the task, and it is responsible for the communication channel required to get information from the server about the task’s state and resources.
An application model with two levels is proposed by them. 
Compared with other migration frameworks, they focus on maximizing usage of local facilities for user tasks.
As examples for applications of ScudOSGi, they investigated two use-cases: smart player and multi-user shopping.

The purpose of \cite{r3-10.1145/2556288.2557199} is to demonstrate the efficiency of multi-engine distributed process execution in an effort to improve adaptability in response to ad-hoc context changes.
They present an abstraction meta-model of migration data that can be used to enhance existing processes with the ability to perform run-time migration.
Both process modelers and initiators can include their intentions and privacy needs in the approach, as well as execute processes sequentially or in parallel.
In addition to the use of physical process fragmentation, a concept has been proposed for realizing logical fragmentation based on the guidelines of process migration.
Process migration is more flexible than physical fragmentation because it provides the flexibility of distributing running process instances at run-time while respecting the process modeler’s guidelines.
Additionally, the papers address privacy and security issues in an explicit manner.

These researches focus on different aspects of run-time migration, while our approach focuses on supporting actual run-time state migration.
