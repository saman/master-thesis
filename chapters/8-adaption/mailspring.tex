
\paragraph{Library Usage}
The main process's script of Mailspring is implemented in \lstinline[basicstyle=\ttfamily]{application.ts} which is in \lstinline[basicstyle=\ttfamily]{app/src/browser/} directory.
We have imported the JavaScript library of run-time state migration, \textit{search} and \textit{sending-email} Application State Models and their interfaces in this script. Listing \ref{lis:js-import} shows how this import is done.

\FloatBarrier
\begin{code}
\begin{js2}
import RuntimeStateMigration from 'rsm-node';
import sendingEmail from '../models/sending-email.json';
import searchEmail from '../models/search.json';
import { 
    SearchObject,
    Convert as SearchObjectConvert
} from '../models/Search';
import {
    SendingEmailObject,
    Convert as SendingEmailObjectConvert
} from '../models/SendingEmail';
\end{js2}
\caption{Mailspring Adaption: Import of JavaScript library of run-time state migration, Application State Models and their interfaces}
\label{lis:js-import}
\end{code}
\FloatBarrier

\paragraph{Application Initialization}
The main process's script has a entry point which is \lstinline[basicstyle=\ttfamily]{start} method. An instance of JavaScript library has been made in this method with some configuration parameters. Also, callback methods which we have implemented in main process's script are bound to the instance of the library as parameters. Furthermore, 
Application State Models are added in this method with \textit{addModel} method. Moreover, \textit{DeviceIntroduction} method is being call here. Listing \ref{lis:js-app-init} shows a snippet that we use to implement the instruction above.

\FloatBarrier
\begin{code}
\begin{js2}
this.rsm = new RuntimeStateMigration(
  { 
    name: 'Mailspring MAC',
    server: {
        url: 'mqtt://130.185.123.111',
        port: 1883
    }
  },
  this.rsmOnStateRequest.bind(this),
  this.rsmOnStateReceive.bind(this),
  this.rsmOnStateMigration.bind(this),
  this.rsmOnDeviceJoin.bind(this),
  this.rsmOnDeviceLeave.bind(this),
);

this.rsm.addModel(sendingEmail);
this.rsm.addModel(searchEmail);
this.rsm.introduce();
\end{js2}
\caption{Mailspring Adaption: Application initialization necessary codes}
\label{lis:js-app-init}
\end{code}
\FloatBarrier

\paragraph{Transferring Run-time State}
In case of any changing in inputs of \textit{search} and \textit{compose} views in Mailspring, the current run-time state of them is being stored with \textit{setState} in library, in case of an application request for them. Also, other devices with the same Application State Model should be noticed whether this device has a state or not with \textit{setHasState} method. Listing \ref{lis:js-setate} shows how we used this method for \textit{search} state.

\FloatBarrier
\begin{code}
\begin{js2}
const model_name = 'search';

this.rsm.setHasState(model_name, true);

const search: SearchObject = state;
this.rsm.setState(model_name, search);
\end{js2}
\caption{Mailspring Adaption: Using the setState and setHasState methods}
\label{lis:js-setate}
\end{code}
\FloatBarrier


As we discussed in Chapter \ref{ch:implementation}, when any device request for run-time state of the source application, the \textit{onStateRequest} callback will be called. Listing \ref{lis:js-request} shows how we have implemented this method in Mailspring. 
  
\FloatBarrier
\begin{code}
\begin{js2}
rsmOnStateRequest(data) {
   this.rsm.sendState(data.model_name, data.device.id);
}
\end{js2}
\caption{Mailspring Adaption: Using the onStateRequest method}
\label{lis:js-request}
\end{code}
\FloatBarrier

When a device receives a state the \textit{onStateReceive} callback will be called. In Mailspring we wrote a code in \textit{rsmOnStateReceive} method to find the corresponding window by and send the run-time state to UI with Electron \lstinline[basicstyle=\ttfamily]{ipcRenderer} by \textit{transferToWindow} method. In UI components we implemented a method to adjust the new run-time state. Listing \ref{lis:js-receive} shows a method that adjust the new run-time state of \textit{search} state.

\FloatBarrier
\begin{code}
\begin{js2}

rsmOnStateReceive(data) {
    this.transferToWindow(data.model_name, data.state);
}

// UI Code Example
_onState = (data, wndwKey) => {
    const search: SearchObject = SearchObjectConvert.toSearch(data);
    
    // Mailspring native variable and methods that we have changed
    this._searchQuery = search.query || "";
    this._submit = search.submit || false;
    this.trigger();
};
\end{js2}
\caption{Mailspring Adaption: Using the onStateReceive method to adjust new run-time state}
\label{lis:js-receive}
\end{code}
\FloatBarrier

After finishing the migration, the target device should notify the source device about finalizing the run-time state migration with \textit{setMigration} method. In Mailspring we changed the UI to call \textit{setMigration} method when the run-time state adjusted. Listing \ref{lis:js-mg-set} shows this implementation for \textit{search} state.
Also, when the source device receives the migration message and \textit{onStateMigration} callback is being called, its the run-time state should be reset. Listing \ref{lis:js-mg-get} shows how we implement this callback. 

\FloatBarrier
\begin{code}
\begin{js2}
const model_name = 'search';
this.rsm.setMigration(model_name, source_device_id);
\end{js2}
\caption{Mailspring Adaption: Notifying source device the migration is complete}
\label{lis:js-mg-set}
\end{code}
\FloatBarrier

\FloatBarrier
\begin{code}
\begin{js2}
rsmOnStateMigration(data) {
    const search: SearchObject = {};
    this.rsm.setState(data.model_name, SearchObject);
    this.rsm.setHasState(data.model_name, false);
}
\end{js2}
\caption{Mailspring Adaption: Resetting the run-time state when receives a migration message}
\label{lis:js-mg-get}
\end{code}
\FloatBarrier