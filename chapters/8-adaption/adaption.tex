\chapter{Adaption of Example Applications}
\label{ch:adaption}
In this chapter, to show the feasibility of the run-time state migration approach, two open-source e-mail clients (Mailspring and K-9 Mail) adapted to support run-time state migration based on the developed approach as part of the thesis.

As a developer who wants to implement the enabling run-time state migration, we used the developed approach, middleware, libraries, deriving interfaces, helper tools, and adaptability to existing same-purpose applications. 

\section{Application State Models}
After analyzing states of Mailspring and K-9 Mail applications and their source code (Table \ref{tab:states_of_email_applications}), we decided to model \textit{search} and \textit{sending-email} states. We have developed two common Application State Models in Chapter \ref{ch:language} for these two applications which are Listing \ref{lis:search-schema} and \ref{lis:sending-email-schema}. In this chapter, as a developer we add the support of these two Application State Models to Mailspring and K-9 Mail.



\section{Example Applications}
Suggestions for e-mail clients to be adapted are K-9 Mail for Android and Mailspring for desktop operating systems.
These applications are for different platforms, and their source code is freely accessible to allow the integration of libraries. 
 They are interactive applications for end-users and have sufficient complexity (e.g., applications do not have only one single state).
K-9 Mail Android application is developed in Java and Kotlin; therefore, it requires the Android library, which we developed for run-time state migration. Moreover, Mailspring is developed using Electron and TypeScript for desktop operating systems, and it needs the JavaScript library for run-time state migration.

\subsection{Mailspring}
Mailspring is an open-source e-mail client application\footnote{\url{https://getmailspring.com/}}. The source code of this application is available on GitHub\footnote{\url{https://github.com/Foundry376/Mailspring}}. This application can be installed on desktop operating systems like macOS, Linux, and Windows.
\subsubsection{Architecture}
Because Mailspring is written in TypeScript (which is a superset of JavaScript) and has been built on top of Electron, its main script which specified in \lstinline[basicstyle=\ttfamily]{package.json}, is referred to as the main process. The main process's script can display a UI by generating web pages. Only one main process is involved in an Electron application. Also, Electron has another type of process, which is the renderer process, and each web page runs in its own renderer process. Electron provides a special API for communication of main process and renderer processes \cite{electron}. Furthermore, Mailspring's UI is developed in ReactJS, which is a JavaScript UI library. Each part of the UI is a ReactJS component.

\subsubsection{Adaptions}
The run-time state migration JavaScript library and Application State Models interfaces are integrated into the main process's script. On the other hand, all UI adaptions are in ReactJS components. Adaptions are added to the fork of this project, hosted on GitHub\footnote{\url{https://github.com/asml-lang/mailspring}}. 

Following is a brief description of the main implementation parts of in Mailspring.

\paragraph{Library Usage}
The main process's script of Mailspring is implemented in \lstinline[basicstyle=\ttfamily]{application.ts} which is in \lstinline[basicstyle=\ttfamily]{app/src/browser/} directory.
We have imported the JavaScript library of run-time state migration, \textit{search} and \textit{sending-email} Application State Models and their interfaces in this script. Listing \ref{lis:js-import} shows how this import is done.

\FloatBarrier
\begin{code}
\begin{js2}
import RuntimeStateMigration from 'rsm-node';
import sendingEmail from '../models/sending-email.json';
import searchEmail from '../models/search.json';
import { 
    SearchObject,
    Convert as SearchObjectConvert
} from '../models/Search';
import {
    SendingEmailObject,
    Convert as SendingEmailObjectConvert
} from '../models/SendingEmail';
\end{js2}
\caption{Mailspring Adaption: Import of JavaScript library of run-time state migration, Application State Models and their interfaces}
\label{lis:js-import}
\end{code}
\FloatBarrier

\paragraph{Application Initialization}
The main process's script has a entry point which is \lstinline[basicstyle=\ttfamily]{start} method. An instance of JavaScript library has been made in this method with some configuration parameters. Also, callback methods which we have implemented in main process's script are bound to the instance of the library as parameters. Furthermore, 
Application State Models are added in this method with \textit{addModel} method. Moreover, \textit{DeviceIntroduction} method is being call here. Listing \ref{lis:js-app-init} shows a snippet that we use to implement the instruction above.

\FloatBarrier
\begin{code}
\begin{js2}
this.rsm = new RuntimeStateMigration(
  { 
    name: 'Mailspring MAC',
    server: {
        url: 'mqtt://130.185.123.111',
        port: 1883
    }
  },
  this.rsmOnStateRequest.bind(this),
  this.rsmOnStateReceive.bind(this),
  this.rsmOnStateMigration.bind(this),
  this.rsmOnDeviceJoin.bind(this),
  this.rsmOnDeviceLeave.bind(this),
);

this.rsm.addModel(sendingEmail);
this.rsm.addModel(searchEmail);
this.rsm.introduce();
\end{js2}
\caption{Mailspring Adaption: Application initialization necessary codes}
\label{lis:js-app-init}
\end{code}
\FloatBarrier

\paragraph{Transferring Run-time State}
In case of any changing in inputs of \textit{search} and \textit{compose} views in Mailspring, the current run-time state of them is being stored with \textit{setState} in library, in case of an application request for them. Also, other devices with the same Application State Model should be noticed whether this device has a state or not with \textit{setHasState} method. Listing \ref{lis:js-setate} shows how we used this method for \textit{search} state.

\FloatBarrier
\begin{code}
\begin{js2}
const model_name = 'search';

this.rsm.setHasState(model_name, true);

const search: SearchObject = state;
this.rsm.setState(model_name, search);
\end{js2}
\caption{Mailspring Adaption: Using the setState and setHasState methods}
\label{lis:js-setate}
\end{code}
\FloatBarrier


As we discussed in Chapter \ref{ch:implementation}, when any device request for run-time state of the source application, the \textit{onStateRequest} callback will be called. Listing \ref{lis:js-request} shows how we have implemented this method in Mailspring. 
  
\FloatBarrier
\begin{code}
\begin{js2}
rsmOnStateRequest(data) {
   this.rsm.sendState(data.model_name, data.device.id);
}
\end{js2}
\caption{Mailspring Adaption: Using the onStateRequest method}
\label{lis:js-request}
\end{code}
\FloatBarrier

When a device receives a state the \textit{onStateReceive} callback will be called. In Mailspring we wrote a code in \textit{rsmOnStateReceive} method to find the corresponding window by and send the run-time state to UI with Electron \lstinline[basicstyle=\ttfamily]{ipcRenderer} by \textit{transferToWindow} method. In UI components we implemented a method to adjust the new run-time state. Listing \ref{lis:js-receive} shows a method that adjust the new run-time state of \textit{search} state.

\FloatBarrier
\begin{code}
\begin{js2}

rsmOnStateReceive(data) {
    this.transferToWindow(data.model_name, data.state);
}

// UI Code Example
_onState = (data, wndwKey) => {
    const search: SearchObject = SearchObjectConvert.toSearch(data);
    
    // Mailspring native variable and methods that we have changed
    this._searchQuery = search.query || "";
    this._submit = search.submit || false;
    this.trigger();
};
\end{js2}
\caption{Mailspring Adaption: Using the onStateReceive method to adjust new run-time state}
\label{lis:js-receive}
\end{code}
\FloatBarrier

After finishing the migration, the target device should notify the source device about finalizing the run-time state migration with \textit{setMigration} method. In Mailspring we changed the UI to call \textit{setMigration} method when the run-time state adjusted. Listing \ref{lis:js-mg-set} shows this implementation for \textit{search} state.
Also, when the source device receives the migration message and \textit{onStateMigration} callback is being called, its the run-time state should be reset. Listing \ref{lis:js-mg-get} shows how we implement this callback. 

\FloatBarrier
\begin{code}
\begin{js2}
const model_name = 'search';
this.rsm.setMigration(model_name, source_device_id);
\end{js2}
\caption{Mailspring Adaption: Notifying source device the migration is complete}
\label{lis:js-mg-set}
\end{code}
\FloatBarrier

\FloatBarrier
\begin{code}
\begin{js2}
rsmOnStateMigration(data) {
    const search: SearchObject = {};
    this.rsm.setState(data.model_name, SearchObject);
    this.rsm.setHasState(data.model_name, false);
}
\end{js2}
\caption{Mailspring Adaption: Resetting the run-time state when receives a migration message}
\label{lis:js-mg-get}
\end{code}
\FloatBarrier


\subsection{K-9 Mail}
K-9 Mail is an open-source e-mail client application\footnote{\url{https://k9mail.app/}}. The source code of this application is available on GitHub\footnote{\url{https://github.com/k9mail/k-9}}. This application can be installed on Android devices.

\subsubsection{Architecture}
K-9 Mail is written in Java 8. However, the new code base is partially migrated to Kotlin. The K-9 Mail project consists of different modules. The \textit{k9mail} is the main module that includes code for database interaction, notification, and activities. Another module is the \textit{k9mail-library} which is the back-end code for decoding e-mails and contacting mail providers. 

\subsubsection{Adaptions}
As we worked only on the application, all changes are implemented in \textit{k9mail} module. The run-time state migration Android library has been added to this module as a dependency. The \textit{k9mail} consist of different packages. The integration of run-time state migration is implemented in \textit{ui} package. Adaptions are added to the fork of this project, hosted on GitHub\footnote{\url{https://github.com/asml-lang/k-9}}. 

Following is a brief description of the main implementation parts in K-9 Mail.

\paragraph{Library Usage}
We have imported the Java library of run-time state migration, \textit{search} and \textit{sending-email} Application State Models and their interfaces in each corresponding class which are \textit{MessageList.kt} for \textit{search} state and \textit{MessageCompose.java} for \textit{sending-email} state. Listing \ref{lis:java-import} shows how this import is done for \textit{search} state.

\FloatBarrier
\begin{code}
\begin{javas}
import com.github.asml.rsm.android.RuntimeStateMigration
import com.github.asml.rsm.android.interfaces.OnDeviceJoinListener
import com.github.asml.rsm.android.interfaces.OnDeviceLeaveListener
import com.github.asml.rsm.android.interfaces.OnStateMigrationListener
import com.github.asml.rsm.android.interfaces.OnStateReceiveListener
import com.github.asml.rsm.android.interfaces.OnStateRequestListener
import com.fsck.k9.models.SearchState
\end{javas}
\caption{K-9 Mail Adaption: Import of Android library of run-time state migration, Application State Models and their interfaces}
\label{lis:java-import}
\end{code}
\FloatBarrier

\paragraph{Application Initialization}
An instance of Java library has been made in \textit{messageListUiModule} with some configuration. Listing \ref{lis:java-app-init} shows initializing the library.

\FloatBarrier
\begin{code}
\begin{java}
RuntimeStateMigration.init(
        androidApplication(),
        Config(
            Server("tcp://130.185.123.111", 1883),
            "K9-Mail Android"
        )
)
\end{java}
\caption{K-9 Mail Adaption: Application initialization necessary codes}
\label{lis:java-app-init}
\end{code}
\FloatBarrier

Listing \ref{lis:java-app-instance} shows \textit{MessageList} class make an instance of the library on its constructor. Also, callback methods which we have implemented are bound to the instance of the library with their setter methods. 

\FloatBarrier
\begin{code}
\begin{javas}
private val RSM_MODEL = "search"
private RuntimeStateMigration rsm = RuntimeStateMigration.getInstance();

rsm.setOnDeviceJoinListener(this::onDeviceJoin)
rsm.setOnDeviceLeaveListener(this::onDeviceLeave)
rsm.setOnStateRequestListener(this::onStateRequest)
rsm.setOnStateReceiveListener(this::onStateReceive)
rsm.setOnStateMigrationListener(this::onStateMigration)       
\end{javas}
\caption{K-9 Mail Adaption: Instance of Android Library}
\label{lis:java-app-instance}
\end{code}
\FloatBarrier

Each class has a entry point which is \lstinline[basicstyle=\ttfamily]{onCreate} method, which \textit{DeviceIntroduction} method is being called there and Application State Models are added in this method with \textit{addModel} method. Listing \ref{lis:java-app-intro} shows a snippet that we use to implement the instruction above.

\FloatBarrier
\begin{code}
\begin{java}
rsm.addModel(SearchState)
rsm.introduce()
\end{java}
\caption{K-9 Mail Adaption: Using \textit{DeviceIntdocution} in Android Library}
\label{lis:java-app-intro}
\end{code}
\FloatBarrier

\paragraph{Transferring Run-time State}
In case of any changing in inputs of \textit{MessageList} and \textit{MessageCompose} views other devices with the same Application State Model should be noticed whether this device has a state or not with \textit{setHasState} method. Listing \ref{lis:java-setate} shows how we used this method for \textit{search} state.

\FloatBarrier
\begin{code}
\begin{java}
override fun onQueryTextChange(newText: String?): Boolean {
    rsm.setHasState(RSM_MODEL, newText?.isNotEmpty() == true)
    return false
}
\end{java}
\caption{K-9 Mail Adaption: Using the setHasState method}
\label{lis:java-setate}
\end{code}
\FloatBarrier


As we discussed in Chapter \ref{ch:implementation}, when any device request for run-time state of the source application, the \textit{onStateRequest} callback will be called. Listing \ref{lis:java-request} shows how we have implemented this method in K-9 Mail for \textit{search} state. 
  
\FloatBarrier
\begin{code}
\begin{javas}
fun onStateRequest(modelName: String?, device: Device?) {
    if (modelName == RSM_MODEL) {
        rsm.setState(
            RSM_MODEL,
            SearchState(
                searchView.query.toString(),
                submit_value)
            .toString()
        )
        rsm.sendState(RSM_MODEL, device?.id)
    }
}
\end{javas}
\caption{K-9 Mail Adaption: Using the onStateRequest and sendState methods}
\label{lis:java-request}
\end{code}
\FloatBarrier

As we mentioned before, when a device receives a state the \textit{onStateReceive} callback will be called. In K-9 Mail we developed a method to distinguish between the received run-time state, assign it to corresponding view and adjust the new run-time state. Listing \ref{lis:java-receive} shows a method that adjust the new run-time state of \textit{search} state.

\FloatBarrier
\begin{code}
\begin{js2}
val searchState = SearchState.fromJsonString(state)
searchView.setQuery(searchState.query, searchState.isSubmit)
\end{js2}
\caption{K-9 Mail Adaption: Using the onStateReceive method to adjust new run-time state}
\label{lis:java-receive}
\end{code}
\FloatBarrier

Like Mailspring, in K-9 Mail we use \textit{setMigration} method to notify other devices about finalizing the migration. Listing \ref{lis:java-mg-set} shows this implementation for \textit{search} state.
Also,  the implementation of \textit{onStateMigration} callback which resets \textit{search} state in K-9 Mail shown in Listing \ref{lis:java-mg-get}.

\FloatBarrier
\begin{code}
\begin{java}
rsm.setMigration(RSM_MODEL, device?.id)
\end{java}
\caption{K-9 Mail Adaption: Notifying source device the migration is complete}
\label{lis:java-mg-set}
\end{code}
\FloatBarrier

\FloatBarrier
\begin{code}
\begin{java}
if (modelName == RSM_MODEL && searchView.isShown) {
    searchView.setQuery("", false)
}
\end{java}
\caption{K-9 Mail Adaption: Resetting the run-time state when receives a migration message}
\label{lis:java-mg-get}
\end{code}
\FloatBarrier

\subsection{UI Adaptions}
In this section, we explain all implemented UI adaptions in both applications to achieve a proper behavior of run-time state migration. 

\subsubsection{Notifications}
When a device joins or leaves, other devices which are subscribed to the same Application State Model's topic will be notified by \textit{onDeviceJoin} and \textit{onDeviceLeave} callbacks. Devices that gets this message show a native notification on their application and inform the user about the connectivity status of the device. Figures \ref{fig:adapt-noti} shows a notification on Mailspring and K-9 Mail about joining a new device to \textit{search} topic.

\subsubsection{Run-time State Migration Button}
For each view of a state, we developed a floating action button as the main button of run-time state migration. When the user clicks on this button, applications display two other buttons as \textit{Set State} for push method and \textit{Get State} for the pull method. The user can start the migration process by clicking one of these buttons on the view of current run-time state. Figures \ref{fig:adapt-noti} and \ref{fig:adapt-compose} show run-time state migration button on different views of Mailspring and K-9 Mail. 

\subsubsection{Device List Modal Box}
When the user clicks on the run-time state migration button and clicks on one of the migration methods' buttons, applications show a modal box. This modal box displays the name of the current run-time state, migration method, and the list of the devices available for migration by \textit{getDevices}. The user should select a device from the list and click on the \textit{migrate} button. By clicking on the \textit{migrate} button, If the chosen method is push, the run-time state can be migrated by \textit{sendState} method explained in 7.3.1. Otherwise, for the pull method, the run-time state can be requested by \textit{getStateDevice} method explained in 7.3.1. In the end, the modal box gets closed. Figure \ref{fig:adapt-modal} shows device list modal box on Mailspring and K-9 Mail.

\subsubsection{Run-time State Adjustments}
We adjust both applications for both \textit{search} and \textit{sending-email} states. If the target device receives a new run-time state by \textit{onStateReceive} callback explained in 7.3.2, application react accordingly. For \textit{search} state, the input of the query text gets updated with the source application's query text. Moreover, if the search is already submitted on the source application, the target device also tries to submit the query text and shows a result. Moreover, for \textit{sending-email} state, if the target device gets a new run-time state by push method, it displays the new draft on a new compose window. Also, the user can click on the run-time state migration button on the compose window and pull and push the run-time state of the current window. After adjustment, the target device announces the end of migration to the source device with \textit{setMigration} method explained in 7.3.1.

\subsubsection{Removing Run-time State}
When \textit{onStateMigration} callback explained in 7.3.2 gets called, application resets all input of the corresponding run-time state. For \textit{search} state, in query text input will be empty, and the list of results will be reset. For \textit{sending-email}, the current compose window will be closed and the draft will be deleted.

\subsubsection{UI Screenshots}
In this section, we demonstrate UI adjustments in example applications.

Figure \ref{fig:adapt-noti} shows a run-time state migration button which belongs to \textit{search} state. Also, this figure displays notifications for joining K-9 Mail and Mailspring by sharing the \textit{search} state.


\FloatBarrier
\begin{figure}[H]
    \includegraphics[width=\linewidth]{../figures/adapt-noti.png}
    \centering
    \caption{Screenshot of Native Notifications and Run-time State Migration Button}
    \label{fig:adapt-noti}
\end{figure}
\FloatBarrier

Figure \ref{fig:adapt-compose} shows the compose window of Mailspring and K-9 Mail which have a run-time state migration button.

\FloatBarrier
\begin{figure}[H]
    \includegraphics[width=\linewidth]{../figures/adapt-compose.png}
    \centering
    \caption{Screenshot of Compose Window}
    \label{fig:adapt-compose}
\end{figure}
\FloatBarrier

Figure \ref{fig:adapt-modal} shows the device modal of Mailspring and K-9 Mail which have a migrate button.

\FloatBarrier
\begin{figure}[H]
    \includegraphics[width=\linewidth]{../figures/adapt-modal.png}
    \centering
    \caption{Screenshot of Device List Modal Box}
    \label{fig:adapt-modal}
\end{figure}
\FloatBarrier


\subsection{Storyboard}
Figure \ref{fig:storyboard} shows a storyboard of implemented run-time state migration approach on two same-purpose applications which are Mailspring and K-9 Mail.

\paragraph{Shot 1}
Shows only K-9 Mail on an Android device is in the network. The user clicks on compose e-mail button on K-9 Mail to write an e-mail

\paragraph{Shot 2}
A compose form shows up on K-9 Mail and users writes the e-mail. She wants to switch to another device, so she runs Mailspring on a Mac device.

\paragraph{Shot 3}
Mailspring which has the common \textit{sending-email} Application State Model with K-9 Mail, joins the network. When Mailspring joins, K-9 Mail shows a notification to the user. She clicks on run-time state migration button.

\paragraph{Shot 4}
K-9 Mail display migration options which \textit{Set State} is for push method and \textit{Get State} is for pull method. The user clicks on \textit{Set State} as she wants to migrate from K-9 Mail to Mailspring.

\paragraph{Shot 5}
The \textit{Device List Modal} shows up on K-9 Mail and user choose the target application which is \textit{MailSpring Mac} and clicks on \textit{MIGRATE} button.

\paragraph{Shot 6}
The compose form on K-9 Mail gets closed and a compose window appears on Mailspring with the same e-mail that was in the K-9 Mail. The user again wants to migrate the run-time state. She clicks on compose e-mail button on K-9 Mail.

\paragraph{Shot 7}
User changes the e-mail on Mailspring and this time as she wants to migrate from Mailspring to K-9 Mail, clicks on \textit{Get State} .

\paragraph{Shot 8}
The \textit{Device List Modal} shows up on K-9 Mail and user choose the source application which is \textit{MailSpring Mac} and clicks on \textit{MIGRATE} button.

\paragraph{Shot 9}
The compose window on Mailspring gets closed and a compose form appears on K-9 Mail with the same e-mail that user have changed in Mailspring.

\newpage
\FloatBarrier
\begin{sidewaysfigure}

\begin{figure}[H]
    \includegraphics[width=\linewidth]{../figures/migration.png}
    \centering
    \caption{Storyboard of run-time state migration}
    \label{fig:storyboard}
\end{figure}
\end{sidewaysfigure}

\FloatBarrier
