
\paragraph{Library Usage}
We have imported the Java library of run-time state migration, \textit{search} and \textit{sending-email} Application State Models and their interfaces in each corresponding class which are \textit{MessageList.kt} for \textit{search} state and \textit{MessageCompose.java} for \textit{sending-email} state. Listing \ref{lis:java-import} shows how this import is done for \textit{search} state.

\FloatBarrier
\begin{code}
\begin{javas}
import com.github.asml.rsm.android.RuntimeStateMigration
import com.github.asml.rsm.android.interfaces.OnDeviceJoinListener
import com.github.asml.rsm.android.interfaces.OnDeviceLeaveListener
import com.github.asml.rsm.android.interfaces.OnStateMigrationListener
import com.github.asml.rsm.android.interfaces.OnStateReceiveListener
import com.github.asml.rsm.android.interfaces.OnStateRequestListener
import com.fsck.k9.models.SearchState
\end{javas}
\caption{K-9 Mail Adaption: Import of Android library of run-time state migration, Application State Models and their interfaces}
\label{lis:java-import}
\end{code}
\FloatBarrier

\paragraph{Application Initialization}
An instance of Java library has been made in \textit{messageListUiModule} with some configuration. Listing \ref{lis:java-app-init} shows initializing the library.

\FloatBarrier
\begin{code}
\begin{java}
RuntimeStateMigration.init(
        androidApplication(),
        Config(
            Server("tcp://130.185.123.111", 1883),
            "K9-Mail Android"
        )
)
\end{java}
\caption{K-9 Mail Adaption: Application initialization necessary codes}
\label{lis:java-app-init}
\end{code}
\FloatBarrier

Listing \ref{lis:java-app-instance} shows \textit{MessageList} class make an instance of the library on its constructor. Also, callback methods which we have implemented are bound to the instance of the library with their setter methods. 

\FloatBarrier
\begin{code}
\begin{javas}
private val RSM_MODEL = "search"
private RuntimeStateMigration rsm = RuntimeStateMigration.getInstance();

rsm.setOnDeviceJoinListener(this::onDeviceJoin)
rsm.setOnDeviceLeaveListener(this::onDeviceLeave)
rsm.setOnStateRequestListener(this::onStateRequest)
rsm.setOnStateReceiveListener(this::onStateReceive)
rsm.setOnStateMigrationListener(this::onStateMigration)       
\end{javas}
\caption{K-9 Mail Adaption: Instance of Android Library}
\label{lis:java-app-instance}
\end{code}
\FloatBarrier

Each class has a entry point which is \lstinline[basicstyle=\ttfamily]{onCreate} method, which \textit{DeviceIntroduction} method is being called there and Application State Models are added in this method with \textit{addModel} method. Listing \ref{lis:java-app-intro} shows a snippet that we use to implement the instruction above.

\FloatBarrier
\begin{code}
\begin{java}
rsm.addModel(SearchState)
rsm.introduce()
\end{java}
\caption{K-9 Mail Adaption: Using \textit{DeviceIntdocution} in Android Library}
\label{lis:java-app-intro}
\end{code}
\FloatBarrier

\paragraph{Transferring Run-time State}
In case of any changing in inputs of \textit{MessageList} and \textit{MessageCompose} views other devices with the same Application State Model should be noticed whether this device has a state or not with \textit{setHasState} method. Listing \ref{lis:java-setate} shows how we used this method for \textit{search} state.

\FloatBarrier
\begin{code}
\begin{java}
override fun onQueryTextChange(newText: String?): Boolean {
    rsm.setHasState(RSM_MODEL, newText?.isNotEmpty() == true)
    return false
}
\end{java}
\caption{K-9 Mail Adaption: Using the setHasState method}
\label{lis:java-setate}
\end{code}
\FloatBarrier


As we discussed in Chapter \ref{ch:implementation}, when any device request for run-time state of the source application, the \textit{onStateRequest} callback will be called. Listing \ref{lis:java-request} shows how we have implemented this method in K-9 Mail for \textit{search} state. 
  
\FloatBarrier
\begin{code}
\begin{javas}
fun onStateRequest(modelName: String?, device: Device?) {
    if (modelName == RSM_MODEL) {
        rsm.setState(
            RSM_MODEL,
            SearchState(
                searchView.query.toString(),
                submit_value)
            .toString()
        )
        rsm.sendState(RSM_MODEL, device?.id)
    }
}
\end{javas}
\caption{K-9 Mail Adaption: Using the onStateRequest and sendState methods}
\label{lis:java-request}
\end{code}
\FloatBarrier

As we mentioned before, when a device receives a state the \textit{onStateReceive} callback will be called. In K-9 Mail we developed a method to distinguish between the received run-time state, assign it to corresponding view and adjust the new run-time state. Listing \ref{lis:java-receive} shows a method that adjust the new run-time state of \textit{search} state.

\FloatBarrier
\begin{code}
\begin{js2}
val searchState = SearchState.fromJsonString(state)
searchView.setQuery(searchState.query, searchState.isSubmit)
\end{js2}
\caption{K-9 Mail Adaption: Using the onStateReceive method to adjust new run-time state}
\label{lis:java-receive}
\end{code}
\FloatBarrier

Like Mailspring, in K-9 Mail we use \textit{setMigration} method to notify other devices about finalizing the migration. Listing \ref{lis:java-mg-set} shows this implementation for \textit{search} state.
Also,  the implementation of \textit{onStateMigration} callback which resets \textit{search} state in K-9 Mail shown in Listing \ref{lis:java-mg-get}.

\FloatBarrier
\begin{code}
\begin{java}
rsm.setMigration(RSM_MODEL, device?.id)
\end{java}
\caption{K-9 Mail Adaption: Notifying source device the migration is complete}
\label{lis:java-mg-set}
\end{code}
\FloatBarrier

\FloatBarrier
\begin{code}
\begin{java}
if (modelName == RSM_MODEL && searchView.isShown) {
    searchView.setQuery("", false)
}
\end{java}
\caption{K-9 Mail Adaption: Resetting the run-time state when receives a migration message}
\label{lis:java-mg-get}
\end{code}
\FloatBarrier