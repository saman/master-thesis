Based on \textbf{R2 Platform-independent State Specification} and analysis of same-purpose existing applications, we chose to develop a DSL to model the run-time states by specifications. 
Same-purpose application got analyzed and we identified the most common structure a state has and defined them as requirements for a DSL to specify a state as follows.


\subsection{D1 State Specification}
The DSL shall be able to provide the specification of a state by defining a model. State which are following these specifications can be migrated between applications. Also, state specification shall provide a way to define what part of a run-time state is required and what part is optional.

\subsection{D2 Finding Same State Specification}
Same-purpose applications needs to figure out if they are supporting same state specification. To find common models between applications, the DSL must allow specifying a way for finding the same model. These specifications can be a unique model name, model version and keywords. Also, these specifications help developers to find a common models which resolve part of \textbf{R3 Model Repository}.

\subsection{D3 Validating}    
The DSL shall be able to provide a way that a state must be valid according to its specification. Also, In case of improving the DSL for long-term support, the DSL must have a version to allow validating the state specification. So, it can be possible to know the state specification using which version of the DSL.